\begin{multicols}{2}

\section*{Oppgavebeskrivelse}

\subsection*{Introduksjon}

Denne oppgaven er en fortsettelse av tidligere prosjekter med utvikling av et konsept for robotisert vedlikehold utført av en mobil autonom robot. I løpet av tidligere prosjekter er en robotarm, flere sensorer, en trådløs ruter, innebygget batteridrevet strømforsyning og en sentral PC blitt festet til en vogn av aluminium. Vognen står på fire omnihjul med hver sin elektriske motordriver.

\subsection*{Målsetninger for prosjektet}

For å øke robotens grad av autonomi er det ønskelig å utforske mulighetene for pålitelig og trygg forflytning av roboten uten menneskelig påvirkning. Hensikten med en slik forflytning kan være å nå frem til en ladestasjon eller et punkt der en spesiell vedlikeholdsoppgave eller inspeksjon skal utføres, samtidig som hindringer og farlige situasjoner unngås. Som et ledd i å oppfylle disse målsetningene, skal følgende punkter utføres: 

\begin{enumerate}
	\item Utforsk mulige metoder for autonom navigasjon som kan oppfylle målsetningen over, der datasyn er det primære navigasjonshjelpemiddelet.
	
	\item Implementer en, flere eller en kombinasjon av metodene som ble funnet i punkt én. Dette inkluderer installasjon av nytt utstyr som f.eks. nye kameraer om det er nødvendig.
	
	\item Gjør en vurdering av implementasjonens egnethet for autonom navigasjon i testmiljøet (Kontoromgivelser).
	
	\item Vurder hvor godt systemet håndterer feil og potensielt farlige tilstander og situasjoner.
	
	\item Foreslå endringer og forslag til videre arbeid for å forbedre sikkerheten, påliteligheten og egnethet for autonom navigasjon. 
	
	
\end{enumerate}

\columnbreak

\section*{Problem Description}

\subsection*{Introduction}

This project is a continuation of previous projects in developing a concept for robotic maintenance performed by a mobile autonomous robot.  Over the course of previous projects, the system has been equipped with a robot manipulator arm, several sensors, a wireless router, on-board power supply based on batteries and a central PC. This equipment is mounted on an aluminum wagon. The wagon stands on four omni-wheels, each with their own electric motor drivers.

\subsection*{Project Goals}

To increase the robot`s degree of autonomy, it is desired to explore possibilities for reliable and safe movement of the robot without involving a person. The purpose of such a movement may be to reach a docking station or a location where a maintenance or an inspection task will be performed while avoiding obstructions and hazardous situations. As a step towards achieving these goals, the following points shall be carried out:

\begin{enumerate}
	
	\item Explore potential methods for autonomous navigation that may fulfill the goals above, where computer vision is the primary navigational aid. 
	
	\item Implement one, several or a combination of the methods found in point one. This includes selection and installation of new equipment, e.g. cameras, if necessary. 
	
	\item Performance and suitability assessment of the selected implementation with respect to autonomous navigation.
	
	\item Assess how well the system handles errors and potentially hazardous states and situations.
	
	\item Propose changes to the implementation and suggest further work in order to improve the safety and reliability of the system and its ability to navigate autonomously.

	
\end{enumerate}

\end{multicols}